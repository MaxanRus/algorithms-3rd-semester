\begin{center}
 \textbf{2-е занятие по алгоритмам 09-04}
\end{center}

  \textbf{Другое определение потока.} Сеть - это $\langle V, c, s, t \rangle$, где c - это таблица смежности для всех пар вершин(таблица не симметричная).  \textbf{Поток} - это $f : V \times V \to \mathbb{R} $, такая что\break $(1) \forall u, v \in V: f(u, v) \leqslant c(u, v), (2) \forall u, v \in V: f(u, v) = - f(v, u), (3) \forall v \in V \setminus \{s, t\}: \sum\limits_{u\in V}^{}f(u, v)=0 $

  \textbf{Все следующие действия производятся с данным определением потока.} 

  \textbf{Величина потока и доказательство равенства двух определений.} $|f|_s = \sum\limits_{u\in V}^{} f(s, u), |f|_t=\sum\limits_{u\in V}^{} f(u, t)$. $0 = \sum\limits_{u, v \in V}^{} f(u, v) = \sum\limits_{u\in V}^{} f(s, u) - \sum\limits_{u\in V}^{} f(u, t) = |f|_s - |f|_t$.

  \textbf{Разрез.} Пусть есть сеть, тогда разрез - это два множества $(S, T): S \sqcup T = V, s \in S, t \in T$

   \textbf{Чистый поток через разрез.} Пусть есть сеть и разрез, тогда чистый поток через разрез - это $f(S, T) := \sum\limits_{u\in S, v\in T}^{}f(u, v) $

   \textbf{Пропускная способность разреза} - это $c(S, T) = \sum\limits_{u\in S, v \in T}^{} c(u, v)$

   \textbf{Утв.} $f(S, T) \leqslant c(s, T)$

   \textbf{Лемма.}  $\forall (S, T): f(S, T) = |f|$. Доказательство: $|f| = \sum\limits_{u\in S}^{} \sum\limits_{v \in V}^{} f(u, v) + \sum\limits_{u \in T}^{} \sum\limits_{v\in V}^{} f(u, v) =\break= \sum\limits_{u\in S}^{} \sum\limits_{v \in T}^{} f(u, v) = f(S, T)$

   \textbf{Следствие.} $\forall (S, T) |f| \leqslant c(S, T)$

    \textbf{Остаточная сеть} - это $c(u, v) = c(u, v) - f(u, v)$

     \textbf{Теорема.(Форда-Фалкерсона).} Пусть есть сеть, тогда следующие утверждения эквивалентны.
  \begin{enumerate}
    \item f - максимальный поток
    \item В остаточной сети нет пути по не насыщенным ребрам.
    \item $|f|=c(S, T)$ для какого-то разреза.
  \end{enumerate}
  $1\Rightarrow 2$ очевидно. $3 \Rightarrow 1$ очевидно. Доказательство $2 \Rightarrow 3$, например, можно удалить все насыщенные ребра, получится хотя бы две компоненты, очевидным способом выберем разрез.

  \textit{*** Тут гуляют примеры задач, но зачем их записывать? ***}

  Метод \textbf{Форда-Фаркенсона} пока можем находим путь из $s$ в $t$ по ненасыщенным ребрам и пускаем по нему сколько можем. Этот алгоритм верен для целых чисел, но если числа не целые, то алгоритм может \href{https://ru.wikipedia.org/wiki/%D0%90%D0%BB%D0%B3%D0%BE%D1%80%D0%B8%D1%82%D0%BC_%D0%A4%D0%BE%D1%80%D0%B4%D0%B0_%E2%80%94_%D0%A4%D0%B0%D0%BB%D0%BA%D0%B5%D1%80%D1%81%D0%BE%D0%BD%D0%B0#%D0%9F%D1%80%D0%B8%D0%BC%D0%B5%D1%80_%D0%BD%D0%B5_%D1%81%D1%85%D0%BE%D0%B4%D1%8F%D1%89%D0%B5%D0%B3%D0%BE%D1%81%D1%8F_%D0%B0%D0%BB%D0%B3%D0%BE%D1%80%D0%B8%D1%82%D0%BC%D0%B0}{не завершиться}.
  Асимптотика будет $O(|f|(V+E)), O(kE), O(VE)$. Выбирай любую.

  \textbf{Утв.} Пусть в сети ищется поток ФФ, пусть в какой-то момент времени от вершины $v$ до вершины $t$ нет пути, тогда после этого момента такой путь от $v$ до $t$ не появится никогда. Доказательство на пальцах. 

  \textit{*** Тут гуляют примеры задач, но зачем их записывать? ***}
